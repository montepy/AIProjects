\documentclass{article}

\title{Project 1 Answers}
\date{2020-06-29}
\author{Edward Wang, Atishay Mitra, Kyle Love}

\begin{document}
\maketitle

\section{Part 1}
a. Since the agent can only observe its 4 adjacent cells, it will choose to move to the east in the first step since the cell to the east is unblocked and is currently the first step on the shortest path to the target state. After the agent would move to position E3, it would observe that the north and east cells are blocked and adjust its path. 

b. In a finite gridworld, there will be a limited number of cells which the agent must check and therefore the time required to find the path or conclude that a path to the target does not exist will be finite. The time it takes to search is linked to the finite number of cells. In contrast, a search algorithm applied on an infinite, constantly expanding gridworld could take an infinite amount of time to conclude whether an agent can reach its target or not.

The number of moves the agent can make is bounded from above by the number of unblocked cells squared because if an adjacent cell is blocked then the agent will backtrack to the same cell at most one time meaning that they will have visited the cell at most two times within the algorithm before either realizing that there is another path or that there is no path at all to the target. 

\section{Part 2}

\section{Part 3}

\section{Part 4}

\section{Part 5}

\section{Part 6}

\end{document}
